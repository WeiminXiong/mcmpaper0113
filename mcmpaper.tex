%% -----------------------------------
%%
%%
%% Copyright (C)
%%     2022     by latexstudio.net
%%
%%
\documentclass[12pt]{article}


%%=============setting,设置自己的队号和选题============
\gdef\MCMcontrol{3256489}%队号
\newcommand{\problem}{A}%选题

\newcommand{\control}{\MCMcontrol}
\newcommand{\team}{Team \#\ \MCMcontrol}
\newcommand{\headset}{{\the\year}\\MCM/ICM\\Summary Sheet}

%==========定义摘要,摘要的标题可自定义===================
\renewenvironment{abstract}[1]{%
        \small
        \begin{center}%
          {\large\bfseries #1\vspace{-.5em}}%
        \end{center}}
      {}
\newcommand\keywords[1]{%
    \begingroup
    \par
    \noindent\textbf{Keywords:} #1\par
    \endgroup
}

% 目录居中的重定义
\makeatletter
\renewcommand\tableofcontents{%
    \centerline{\normalfont\Large\bfseries\contentsname%
    \@mkboth{%
    \MakeUppercase\contentsname}{\MakeUppercase\contentsname}}%
    \vskip 3ex%
    \@starttoc{toc}%
    \thispagestyle{fancy}
    \clearpage}
\makeatother

\usepackage[toc, page, title, titletoc, header]{appendix}
\usepackage{graphicx}
\graphicspath{{figures/}{img/}}

\usepackage{amsmath,amssymb,amsfonts,amsthm}
\newtheorem{Theorem}{Theorem}[section]
\newtheorem{Lemma}[Theorem]{Lemma}
\newtheorem{Corollary}[Theorem]{Corollary}
\newtheorem{Proposition}[Theorem]{Proposition}
\newtheorem{Definition}[Theorem]{Definition}
\newtheorem{Example}[Theorem]{Example}


%==========设置代码格式===================
\usepackage{xcolor}
\usepackage{listings}
\definecolor{grey}{rgb}{0.8,0.8,0.8}
\definecolor{darkgreen}{rgb}{0,0.3,0}
\definecolor{darkblue}{rgb}{0,0,0.3}
\def\lstbasicfont{\fontfamily{pcr}\selectfont\footnotesize}
\lstset{%
   % numbers=left,
   % numberstyle=\small,%
    showstringspaces=false,
    showspaces=false,%
    tabsize=4,%
    frame=lines,%
    basicstyle={\footnotesize\lstbasicfont},%
    keywordstyle=\color{darkblue}\bfseries,%
    identifierstyle=,%
    commentstyle=\color{darkgreen},%\itshape,%
    stringstyle=\color{black}%
}
\lstloadlanguages{C,C++,Java,Matlab,Mathematica}

\usepackage{geometry}
\geometry{a4paper, margin = 1.2in}

%==========设置页眉格式===================
\usepackage{fancyhdr,lastpage}
\pagestyle{fancy}
\fancyhf{}
\lhead{\small\sffamily \team}
\rhead{\small\sffamily Page \thepage\ of \pageref{LastPage}}
\setlength\parskip{.5\baselineskip}

\usepackage{hyperref}
\usepackage{mathptmx}% newtxtext
\usepackage{lipsum}
\title{The \LaTeX{} Template for MCM Version 1}
\author{\small \href{http://www.latexstudio.net/}
  {\includegraphics[width=7cm]{mcmthesis-logo}}}
\date{\today}
\begin{document}
%==========Summary sheet 格式===================
\thispagestyle{empty}
\begingroup
  \setlength{\parindent}{0pt}
     \begin{minipage}[t]{0.33\linewidth}
     \bfseries\centering%
      Problem Chosen\\[0.7pc]
      {\Huge\textbf{\problem}}\\[2.8pc]
     \end{minipage}%
     \begin{minipage}[t]{0.33\linewidth}
      \centering%
      \textbf{\headset}%
     \end{minipage}%
     \begin{minipage}[t]{0.33\linewidth}
      \centering\bfseries%
       Team Control Number\\[0.7pc]
      {\Huge\textbf{\MCMcontrol}}\\[2.8pc]
     \end{minipage}\par
  \rule{\linewidth}{0.8pt}\par
  %\textbf{\headset}%
  \par
  \endgroup

  \bigskip

\centerline{\Large\bfseries Simple Template  for MCM Contest by latexstudio}

\begin{abstract}{Summary}
\lipsum[1]
\keywords{keyword1; keyword2}
\end{abstract}



 \newpage
 \tableofcontents


%==========设置正文格式===================

\section{Introduction}

\lipsum[2]
\begin{itemize}
\item minimizes the discomfort to the hands, or
\item maximizes the outgoing velocity of the ball.
\end{itemize}
We focus exclusively on the second definition.

\begin{itemize}
\item the initial velocity and rotation of the ball,
\item the initial velocity and rotation of the bat,
\item the relative position and orientation of the bat and ball, and
\item the force over time that the hitter hands applies on the handle.
\end{itemize}
\lipsum[3]
\begin{itemize}
\item the angular velocity of the bat,
\item the velocity of the ball, and
\item the position of impact along the bat.
\end{itemize}
\lipsum[4]
\emph{center of percussion} [Brody 1986], \lipsum[5]

\begin{Theorem} \label{thm:latex}
\LaTeX
\end{Theorem}
\begin{Lemma} \label{thm:tex}
\TeX .
\end{Lemma}
\begin{proof}
The proof of theorem.
\end{proof}

\subsection{Other Assumptions}
\lipsum[6]
\begin{itemize}
\item
\item
\item
\item
\end{itemize}

\lipsum[7]

\section{Analysis of the Problem}
\begin{figure}[h]
\small
\centering
\includegraphics[width=12cm]{mcmthesis-aaa.eps}
\caption{aa} \label{fig:aa}
\end{figure}

\lipsum[8] \eqref{aa}
\begin{equation}
a^2 \label{aa}
\end{equation}

\[
  \begin{pmatrix}{*{20}c}
  {a_{11} } & {a_{12} } & {a_{13} }  \\
  {a_{21} } & {a_{22} } & {a_{23} }  \\
  {a_{31} } & {a_{32} } & {a_{33} }  \\
  \end{pmatrix}
  = \frac{{Opposite}}{{Hypotenuse}}\cos ^{ - 1} \theta \arcsin \theta
\]
\lipsum[9]

\[
  p_{j}=\begin{cases} 0,&\text{if $j$ is odd}\\
  r!\,(-1)^{j/2},&\text{if $j$ is even}
  \end{cases}
\]

\lipsum[10]

\[
  \arcsin \theta  =
  \mathop{{\int\!\!\!\!\!\int\!\!\!\!\!\int}\mkern-31.2mu
  \bigodot}\limits_\varphi
  {\mathop {\lim }\limits_{x \to \infty } \frac{{n!}}{{r!\left( {n - r}
  \right)!}}} \eqno (1)
\]

\section{Calculating and Simplifying the Model  }
\lipsum[11]

\section{The Model Results}
\lipsum[6]

\section{Validating the Model}
\lipsum[9]

\section{Conclusions}
\lipsum[6]

\section{A Summary}
\lipsum[6]

\section{Evaluate of the Mode}

\section{Strengths and weaknesses}
\lipsum[12]

\subsection{Strengths}
\begin{itemize}
\item \textbf{Applies widely}\\
This  system can be used for many types of airplanes, and it also
solves the interference during  the procedure of the boarding
airplane,as described above we can get to the  optimization
boarding time.We also know that all the service is automate.
\item \textbf{Improve the quality of the airport service}\\
Balancing the cost of the cost and the benefit, it will bring in
more convenient  for airport and passengers.It also saves many
human resources for the airline.

\end{itemize}

\begin{thebibliography}{99}
\bibitem{1} D.~E. KNUTH   The \TeX{}book  the American
Mathematical Society and Addison-Wesley
Publishing Company , 1984-1986.
\bibitem{2}Lamport, Leslie,  \LaTeX{}: `` A Document Preparation System '',
Addison-Wesley Publishing Company, 1986.
\bibitem{3}\url{http://www.latexstudio.net/}
\end{thebibliography}

\begin{appendices}

\section{First appendix}

\lipsum[13]

Here are simulation programmes we used in our model as follow.\\

\textbf{\textcolor[rgb]{0.98,0.00,0.00}{Input matlab source:}}
\lstinputlisting[language=Matlab]{./code/mcmthesis-matlab1.m}

\section{Second appendix}

some more text \textcolor[rgb]{0.98,0.00,0.00}{\textbf{Input C++ source:}}
\lstinputlisting[language=C++]{./code/mcmthesis-sudoku.cpp}

\end{appendices}
\end{document}


